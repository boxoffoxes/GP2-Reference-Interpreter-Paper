\section{Implementation}
\label{sec:implementation}
We describe the key components of the reference interpreter with the aim of illustrating the simplicity, clarity, and conciseness of the implementation. A basic knowledge of Haskell is useful but not essential to understand the content in the following sections. 

\begin{figure}[h]
\centering
\begin{tikzpicture} [align=center, arrowout]

\node(parser) at (0,0)  [box, rounded corners] {Parser};

\node(gen) at (3,0) [box, rounded corners] {Transformer};

\node(inter) at (6.5,0) [box, rounded corners] {Interpreter};

\node(apply) at (4.5,-2) [box, rounded corners] {Rule Applier};

\draw[arrowout] (-2, 0.22) -- node[above, text width=1.5cm]{\scriptsize{Host Graph File}} (parser.160);
\draw[arrowout] (-2, -0.22) -- node[below, text width=1.5cm]{\scriptsize{Program File}} (parser.200);
\draw[arrowout] (0,-1.7) -- node[right, text width=1.5cm]{\scriptsize{Rule Application Bound}} (parser.270);

\draw [arrowout] (parser) --  (gen);
\node at (1.35,0.3) {\scriptsize{AST}};

\draw [arrowout] (gen.10) -- node[above, text width=1cm]{\scriptsize{Host Graph}} (inter.167);
\draw [arrowout] (gen.350) --  (inter.193);
\node at (4.9,-0.5) [text width=1cm]{\scriptsize{Program}};

\draw [arrowout] (inter.325) |- (apply.350);
\node at (7.8,-1.3) {\scriptsize{Rule}};

\draw [arrowout] (inter.310) |- (apply.10);
\node at (6.4, -1.3) [text width=1cm]{\scriptsize{Host Graph}};

\draw [arrowout] (apply.90) --  (inter.225);
\node at (4.6,-1) {\scriptsize{Graphs}};

\draw[arrowout] (inter) -- node[above, text width=1cm]{\scriptsize{Output Data}} (8.5,0);

\end{tikzpicture}



\caption{Main data flow of the reference interpreter} \label{fig:architecture}
\end{figure}

\subsection{Overview}
Figure \ref{fig:architecture} shows a data flowchart of the reference interpreter. It takes three inputs: (1) a file containing the textual representation of a GP~2 program, (2) a file containing the textual representation of a host graph, and (3) an upper limit on the number of rule applications to be made before halting program execution. It runs the program on the host graph, traversing either all nondeterministic branches of the program or a single branch, at the behest of the user. The output data is a complete description of all possible outputs. Section \ref{sec:eval} describes the output data in detail.

The interpreter contains approximately 1,000 lines of Haskell source code. Figure \ref{fig:modules} shows the module dependency structure of the interpreter and an indication of module sizes.  

\begin{figure}
\centering
\begin{tikzpicture} [align=center]

\node(main) at (0, 0)  [box] {Main\\Interpreter\\53 lines};
\node(iso) at (2.5, -2.25) [box] {Isomorphism\\Checker\\21 lines};
\node(print) at (2.5, -0.75) [box] {Graph Printer\\34 lines};
\node(eval) at (2.5, 0.75) [box] {Evaluator\\99 lines};
\node(parser) at (2.5, 2.25) [box] {Parser\\230 lines};
\node(apply) at (5, 0.75) [box] {Rule\\Applier\\53 lines};
\node(gmatch) at (7.5, 0.75) [box] {Graph\\Matcher\\43 lines};
\node(graph) at (7.5, -1.5) [box] {Graph\\Library\\76 lines};
\node(lmatch) at (10, -1.5) [box] {Label\\Matcher\\89 lines};
\node(trans) at (10, 0.75) [box] {Checker \&\\Transformer\\118 lines};
\node(libs) at (12.5, -1.5) [box] {Lists \&\\Finite Maps\\60 lines};
\node(ast) at (12.5, 0.75) [box] {AST\\126 lines};

\draw (ast) edge[arrowin, dashed] (trans);
\draw (ast) edge[arrowin, dashed] (lmatch);
\draw (ast.145) edge[arrowin, dashed, bend right=15] (parser);
\draw (libs) edge[arrowin, dashed] (trans);
\draw (libs) edge[arrowin, dashed] (lmatch);
\draw (libs.215) edge[arrowin, dashed, bend left=25] (graph.315);
\draw (trans) edge[arrowin, dashed] (gmatch);
\draw (graph) edge[arrowin, dashed] (gmatch);
\draw (graph) edge[arrowin, dashed] (iso.0);
\draw (graph) edge[arrowin, dashed] (print.0);
\draw (lmatch) edge[arrowin, dashed] (gmatch);
\draw (gmatch.180) edge[arrowin, dashed] (apply);
\draw (apply) edge[arrowin, dashed] (eval.0);
\draw (parser.180) edge[arrowin, dashed] (main);
\draw (iso.180) edge[arrowin, dashed] (main);
\draw (eval.180) edge[arrowin, dashed] (main);
\draw (print.180) edge[arrowin, dashed] (main);

\end{tikzpicture}



\caption{Module dependencies. A module points to any modules on which it depends. Line counts exclude blank lines and comment-only lines} \label{fig:modules}
\end{figure}

\subsection{Parser}
The parser has two components: (1) a host graph parser and (2) a program text parser. Each individual parsing function takes a string as input and attempts to match a prefix of the string to a particular syntactic unit. It uses a library of parser combinators. Their purpose is to neatly compose the parsing functions to cover standard parsing requirements such as alternation and repetition. The parsing code is very similar in appearance to GP~2's context-free grammar: each nonterminal of the grammar is represented by a Haskell function that parses the right-hand side of the grammar rule. For example:

\begin{verbatim}
gpMain :: Parser Main
gpMain = keyword "Main" |> keyword "=" |> pure Main <*> 
         commandSequence
\end{verbatim}

The operators \texttt{|> and <*>} are binary functions: \texttt{|>} ignores the output of its left parser and \texttt{<*>} sequences two parsers. Applications of \texttt{keyword} recognise and discard a string argument, and \linebreak \texttt{commandSequence} is another parsing function. \texttt{Main} is a data constructor for the main node of GP~2's abstract syntax tree.

\subsection{Checking \& Transformation}

The checking and transformation phase extracts semantic information from the AST, such as the types of variables specified in a rule schema's parameter list, and transforms both rule graphs and the host graph into the data structure defined in the graph library. The internal graph representation is a pair of maps from keys to labels for each of nodes and edges separately. Node keys are integers. Edge keys are triples: source key, target key and an integer. Node and edge labels are encoded into the node and edge data types. Operations on graphs are concisely represented using Haskell functions from the Haskell library \texttt{Data.Map} which implements maps efficiently as balanced binary trees. Node and edge enumeration functions also support the use of Haskell's strong list-processing. See Section \ref{sec:graph-match} for details.

\subsection{Label Matching}
The label matching algorithm establishes whether a label from a rule's left-hand side can be matched with a label from the host graph. It takes as input the current \textit{environment}, the set of bindings for label variables, and the two labels to be compared. 

GP~2 labels consist of a mark and a list. The marks are encoded as an abstract data type and are directly comparable. GP~2's lists are naturally encoded as Haskell lists, where each element is a GP~2 atom. Atoms occurring in the host graph are constants (integers, characters or strings), while rule atoms are either constants, variables or a concatenated string\footnote{Expressions and degree operators are forbidden in LHS labels to prevent ambiguous matching.}. If a match binds a variable, the binding must define a compatible extension of the environment.

When comparing atoms, the interesting case occurs if a list variable is encountered. GP~2 allows at most one list variable in any label expression on a left-hand side. This restriction allows binding to host-label segments of determined length, by comparing the lengths of the remainder of the rule label and the remainder of the host label. Matching fails if too few host atoms remain.

\subsection{Graph Matching}\label{sec:graph-match}

Given a rule graph $L$ and a host graph $G$, the graph matcher lazily constructs a list of \texttt{GraphMorphisms}. A \texttt{GraphMorphism} is a data structure containing an environment, a mapping between nodes in $L$ and the corresponding nodes in $G$, and a similar edge map. We use association lists to represent these small mappings, for simplicity and amenability to list-processing.  Morphisms are generated in two stages. First the candidate \texttt{NodeMorphisms} are identified, where a \texttt{NodeMorphism} is an environment and a node mapping. For each such \texttt{NodeMorphism}, the matcher searches for compatible edge mappings and environment extensions to form a set of complete \texttt{GraphMorphisms}.

\vspace{.5\baselineskip}
\noindent
\emph{Node matching.}
For each node $l_k \in L$, the matcher constructs the list of all host nodes \texttt{[$h_{k_1}, \ldots, h_{k_m}$]} that match $l_k$ with respect to label matching and rootedness\footnote{Graphs can be augmented with root nodes to reduce the search space. GP~2's semantics requires that a root node in $L$ must only match a root node in $G$ \cite{Bak-Plump12a}.} An environment is paired with each host node. The result is a list of lists \texttt{[[$h_{1_1}, \ldots, h_{1_m}$],\ldots,[$h_{n_1}, \ldots, h_{n_m}$]]} where $n$ is the number of nodes in $L$. A candidate node mapping is found by injectively selecting one item from each list. The final step is to test each candidate mapping for compatibility with respect to its environment. Haskell's list comprehensions are perfectly suited for this task: the list of lists is computed with a single nested list comprehension, while a second list comprehension is responsible for collating the valid candidate mappings. 

\vspace{.5\baselineskip}
\noindent
\emph{Edge matching.} 
For each edge in $L$, we use a candidate node morphism to determine the required source and target for a corresponding edge in the host graph. The list of candidate host edges is the list of host edges from that source to that target. Each rule edge is checked against each candidate host edge for label compatibility, supported by the environment passed from the node morphism.

\subsection{Rule Application}
Each of the \texttt{GraphMorphisms} produced by the graph matcher is checked against a \emph{dangling condition} and any rule conditions. If these checks succeed, the rule application is performed in the following steps: delete edges, delete nodes, relabel nodes, add nodes, relabel edges, add edges. For relabelling, variables take their values from a \texttt{GraphMorphism}'s environment. 

The dangling condition can be elegantly expressed as follows.
\begin{verbatim}
danglingCondition :: HostGraph -> EdgeMatches -> [NodeId] -> Bool
danglingCondition h ems delns = 
         null [e | hn <- delns, e <- incidentEdges h hn \\ rng ems]
\end{verbatim}

The second argument is an edge map, obtained from a \texttt{GraphMorphism}. The third argument is the set of nodes deleted by the rule. The function body specifies that no host edge \ttt{e} incident to any deleted node \ttt{n} may lie outside of the range of the edge map \ttt{ems}.

\subsection{The Evaluator}\label{sec:eval}
The evaluator applies a GP~2 program to a host graph, subject to an upper bound on the number of rule applications. Often the same graph can be reached through several distinct computational branches. Therefore, when program execution is complete, an isomorphism checker is used to collate the list of output graphs into its isomorphism classes. The output is as follows:

\begin{enumerate}
\item A list of unique output graphs, up to isomorphism, with a count of how many isomorphic copies of each graph were generated.
\item The number of failures. For example, a failure occurs in some contexts if none of a set of rules can be applied to a graph.
\item The number of unfinished computations. A computation is unfinished if the bound on rule applications is reached before the end of the main command sequence.
\end{enumerate}

During program execution the evaluator maintains a list of \texttt{GraphStates}, one for each nondeterministic branch of the computation so far. A \texttt{GraphState} is one of: (1) a graph with its rule application count, (2) a failure symbol with its rule application count, and (3) an unfinished symbol. Each GP~2 control construct is evaluated by a function that takes as input a single \texttt{GraphState} and some program data, returning a list of \texttt{GraphStates}. Only the application of a rule can yield a \texttt{GraphState} with a changed graph.
The rule application process is the workhorse of the interpreter, so here by way of illustration is the
top-level defining equation for the evaluation of a rule-call command:

\begin{verbatim}
evalSimpleCommand max ds (RuleCall rs) (GS g rc) = 
   if rc == max then [Unfinished]
   else case [h | r <- rs, h <- applyRule g $ ruleLookup r ds] of
        [] -> [Failure rc]
        hs -> [GS h (rc+1) | h <- hs]
\end{verbatim}

Here \texttt{max} is the rule application bound, \texttt{ds} is a list of the rule and procedure declarations in the GP~2 program, \texttt{rs} is a list of rules, and \texttt{GS g rc} is the current graph state. \texttt{GS} is the \texttt{GraphState} constructor, \texttt{g} is the working host graph, and \texttt{rc} is the number of rules that have been applied to \texttt{g}. The case-subject list comprehension can be read as, ``for all rules \texttt{r} in \texttt{rs}, apply \texttt{r} to \texttt{g} and produce the list of all output graphs \texttt{h}.'' Each individual rule application may produce multiple output graphs; the list comprehension gathers every possible output into a single lazily-computed list. If \texttt{resultGraphs} is empty, then no rule in \texttt{rs} was applicable, and the list containing the single \texttt{GraphState Failure} is returned. Otherwise, the output graphs are placed into a fresh list of \texttt{GraphStates}, each with an incremented rule-application count.


