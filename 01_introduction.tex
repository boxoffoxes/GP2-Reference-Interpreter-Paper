\section{Introduction}

GP 2 is an experimental programming language in which the major part of
the computational state is a labelled directed graph, and the basic
units by which computational progress is made are subgraph-rewriting
rules.
Choices of rules and subgraphs are non-determinstic, and some of
the control structures above the level of rules involve back-tracking.

The implementation of such a programming language poses some
interesting challenges and opportunities.
Our ultimate goal is to produce a compiler from GP 2 to
high-performance executable code.
This paper reports a first stage towards that goal, the development
of a \emph{reference interpreter} for GP 2.
By this we mean an interpreter written with the main aim of
being clear, concise and correct.
Where there are design choices, simplicity of
definition takes priority over other considerations
such as performance and the richness of functionality.

Section~\ref{sec:gp2language} outlines and illustrates the GP 2 language
graph-programming language.
Section~\ref{sec:benchmark} presents a small set of test programs
written in GP 2.
Section~\ref{sec:usesrequirements} considers the uses and requirements
for a reference interpreter.
Section~\ref{sec:implementation} describes our reference interpreter for
GP 2.
Section~\ref{sec:results} sets out the results of using the reference
interpreter to evaluate the test programs in Section~\ref{sec:benchmark},
including various measures of the sizes of computations and the
resources needed for them.
Section~\ref{sec:conclusionsfuture} offers conclusions and indicates
likely lines of future work.

