% ENVIRONMENTS --------------------------

\theoremstyle{remark}
\newtheorem{example}{Example}


% MISC ----------------------------------

\newcommand{\red}[1]{\textcolor{red}{#1}}
\newcommand{\blue}[1]{\textcolor{blue}{#1}}
\newcommand{\ttt}{\texttt}
\newcommand{\csp}{\hspace{.2em}}


% FORMULAS ------------------------------------

\newcommand{\mylabel}[1]{ \left\{
 \begin{array}{@{}l@{}} #1 \end{array} \right\}
}

\newcommand{\lvar}[1]{\mathtt{#1}}

\newcommand{\Z}{\mathbb{Z}}
\newcommand{\N}{\mathbb{N}}

\newcommand{\mtt}{\mathtt}
\newcommand{\mrm}{\mathrm}
\newcommand{\msf}{\mathsf}

\newcommand{\C}{\mathrm{Char}}
\newcommand{\Adr}{\mathrm{Adr}}
\newcommand{\Prog}{\mathrm{P}}

\newcommand{\fail}{\mathrm{fail}}
\newcommand{\g}{\mathrm{gra}}
\newcommand{\reg}{\mathrm{reg}}
\newcommand{\s}{\mathrm{s}}

\newcommand{\R}{\mathcal{R}}
\renewcommand{\L}{\mathcal{L}}
\newcommand{\I}{\mathcal{I}}
\newcommand{\G}{\mathcal{G}}
\newcommand{\A}{\mathcal{A}}
\renewcommand{\S}{\mathcal{S}}

\newcommand{\sos}{\mathrm{sos}}
\newcommand{\Sem}[1]{\llbracket{#1}\rrbracket}
\newcommand{\trans}[1]{\tau\llbracket{#1}\rrbracket}
\newcommand{\Gbot}{\mathcal{G}_{\bot}}

\newcommand{\X}{\mathrm{X}}
\newcommand{\V}{\mathrm{V}}
\newcommand{\E}{\mathrm{E}}
\newcommand{\D}{\mathrm{D}}
\newcommand{\tuple}[1]{\langle#1\rangle}
\newcommand{\la}{\langle}
\newcommand{\ra}{\rangle}
\newcommand{\dom}{\mathrm{Dom}}

\newcommand{\x}{\mathtt{x}}
\newcommand{\y}{\mathtt{y}}
\newcommand{\z}{\mathtt{z}}

\newcommand{\ifte}[3]{\mathtt{if}\ #1\ \mathtt{then}\ #2\ \mathtt{else}\ #3}
\newcommand{\ift}[2]{\mathtt{if}\ #1\ \mathtt{then}\ #2}
\newcommand{\tryt}[2]{\mathtt{try}\ #1\ \mathtt{then}\ #2}
\newcommand{\tryte}[3]{\mathtt{try}\ #1\ \mathtt{then}\ #2\ \mathtt{else}\ #3}
\newcommand{\noteq}{\mathrel{\text{\texttt{!=}}}}

\renewcommand{\bar}[1]{\overline{#1}}
\newcommand{\wt}[1]{\widetilde{#1}}
\newcommand{\wh}[1]{\widehat{#1}}

\newcommand{\ul}[1]{\underline{#1}}


% REWRITE RELATIONS ---------------------------

% Abstract reduction and term rewriting -------

\newcommand{\symto}{\leftrightarrow}

\newcommand{\DSto}{\mathop{\to}\limits}
\newcommand{\DSsymto}{\mathop{\symto}\limits}

\newcommand{\symredu}{\Leftrightarrow_{\R}}

\newcommand{\dder}{\Rightarrow}
\newcommand{\ldder}{\Leftarrow}
\newcommand{\der}{\Rightarrow^*}
\newcommand{\lder}{\Leftarrow^*}

\newcommand{\DSdder}{\mathop{\Rightarrow}\limits}
\newcommand{\DSldder}{\mathop{\Leftarrow}\limits}
\newcommand{\DSder}{\DSdder^*}
\newcommand{\DSlder}{\DSldder^*}

\newcommand{\DSlongdder}{\mathop{\Longrightarrow}\limits}
\newcommand{\DSllongdder}{\mathop{\Longleftarrow}\limits}


% GRAPHS -------------------------------------

\newcommand{\bluenodecolour}{\graphnodecolour(0,.6,1)}
\newcommand{\rednodecolour}{\graphnodecolour(1,.03,.03)}
\newcommand{\greennodecolour}{\graphnodecolour(0,1,0)}
\newcommand{\grey}{\graphnodecolour{.75}}
\newcommand{\dashed}{\graphlinedash{4 2}}

\newcommand{\script}{\scriptsize}
\newcommand{\fs}{\footnotesize}
\newcommand{\op}[3]{\textnode{#1}(#2){$\mathtt{#3}$}[\graphlinecolour{1}]}
\newcommand{\smallnode}{\graphnodesize{0.30}}
\newcommand{\nodegraph}[1]{% 
 \graphlinewidth{.01}
 \graphlinecolour{0}
 \graphnodecolour{1}
 \graphnodesize{.4}
 \begin{graph}(.4,.2)
  \roundnode{n}(.2,.1)\autonodetext{n}{#1}
 \end{graph}
}
\newcommand{\edgegraph}{% 
 \graphlinewidth{.01}
 \graphlinecolour{0}
 \graphnodecolour{1}
 \graphnodesize{.2}
 \grapharrowlength{.15}
 \grapharrowwidth{.45}
 \begin{graph}(1,.2)
  \roundnode{s}(.2,.1)\autonodetext{s}[s]{\tiny $s$}
  \roundnode{t}(.8,.1)\autonodetext{t}[s]{\tiny $t$}
  \diredge{s}{t}
 \end{graph}
}
\newcommand{\parrule}{% 
 \graphlinewidth{.01}
 \graphlinecolour{0}
 \graphnodecolour{1}
 \graphnodesize{.2}
 \grapharrowlength{.15}
 \grapharrowwidth{.45}
 \begin{graph}(.8,.2)
  \freetext(.6,0){par:}
 \end{graph}
 \begin{graph}(1,.2)
  \roundnode{s}(.2,.1)\autonodetext{s}[s]{\tiny 1}
  \roundnode{t}(.8,.1)\autonodetext{t}[s]{\tiny 2}
  \dirbow{s}{t}{.1}
  \dirbow{s}{t}{-.1}
 \end{graph}
 \begin{graph}(.4,.2)
  \freetext(.2,.1){\small $\dder$}
 \end{graph}
 \begin{graph}(1,.2)
  \roundnode{s}(.2,.1)\autonodetext{s}[s]{\tiny 1}
  \roundnode{t}(.8,.1)\autonodetext{t}[s]{\tiny 2}
  \diredge{s}{t}
 \end{graph}
}
\newcommand{\seqrule}{% 
 \graphlinewidth{.01}
 \graphlinecolour{0}
 \graphnodecolour{1}
 \graphnodesize{.2}
 \grapharrowlength{.15}
 \grapharrowwidth{.45}
 \begin{graph}(.8,.2)
  \freetext(.6,0){seq:}
 \end{graph}
 \begin{graph}(1.6,.2)
  \roundnode{s}(.2,.1)\autonodetext{s}[s]{\tiny 1}
  \roundnode{d}(.8,.1)
  \roundnode{t}(1.4,.1)\autonodetext{t}[s]{\tiny 2}
  \diredge{s}{d}
  \diredge{d}{t}
 \end{graph}
 \begin{graph}(.4,.2)
  \freetext(.2,.1){\small $\dder$}
 \end{graph}
 \begin{graph}(1,.2)
  \roundnode{s}(.2,.1)\autonodetext{s}[s]{\tiny 1}
  \roundnode{t}(.8,.1)\autonodetext{t}[s]{\tiny 2}
  \diredge{s}{t}
 \end{graph}
}
