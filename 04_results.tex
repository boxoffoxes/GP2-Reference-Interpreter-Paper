\section{Performance Evaluation}\label{sec:performanceevaluation}

While not tuned for speed, our reference interpreter still needs to be fast enough to be capable of useful work. In this section we will look at how efficiently it executes the benchmark programs we have described, and what factors affect performance.

% TODO: mention performance of previous versions, and how increased performance brought bugs and limitations to light.


\subsubsection*{Execution environment}

All benchmarking was done on a quad-core Intel i7 clocked at 3.4GHz, with 8GB RAM, running 64-bit Ubuntu 14.04 LTS with kernel 3.13.0.

The GP~2 interpreter is single-threaded, so the number of processor cores should not have a significant impact on the performance.


\subsubsection*{Compilation options}

The interpreter was compiled with GHC version 7.6.3 with optimisations and profiling support enabled, using the following command-line:

\begin{verbatim}
$ ghc -O2 -prof -fprof-auto -rtsopts -o gp2 Main.hs
\end{verbatim}

Profiling information generated by the resulting instrumented binary was used to obtain the numbers presented here, and also to inform the discussion of these results in the next section.


\subsubsection*{Running GP~2}

The \texttt{gp2} executable has three mandatory commandline arguments: a GP~2 program, a host graph on which to run the program, and a upper bound on the number of allowed rule applications.

The rule application bound, $k$ can be used to restrict the size of the search space, only allowing output graphs that can be reached in fewer than $k$ graph transformations. As part of its output, the interpreter reports the number of unfinished computations which halted upon reaching the bound.

% explain commandline options to interpreter (e.g. max apps)

Additionally \texttt{gp2} accepts two optional arguments that control the number of output graphs generated and use of the built-in isomorphism checker:

\begin{description}
	\item[\texttt{--no-iso} \textit{[n]}]  Disable the isomorphism checker. As this can result in a very large number of output graphs, this flag accepts an optional numeric parameter which sets a maximum number of graphs to generate.

	\item[\texttt{--one}] Only produce a single output graph. Equivalent to \texttt{--no-iso 1}.
\end{description}

Benchmarks were run using the following commandline. We limited execution time to 5 minutes per program/host graph pair. The rule application bound was set sufficiently high that the limit is not reached in any of the benchmarks being run.

\begin{verbatim}
$ timeout --foreground 5m time \
      gp2 +RTS -p -sgc.prof -RTS $GPOPT $PROG $GRAPH 1000000
\end{verbatim}

Each benchmark was run twice: once with \texttt{\$GPOPT} set to \texttt{--one} (single-output graph mode) and once with it unset (all possible results mode).

An unfortunate property of the Haskell memory profiler is that memory profiling information is not saved to disk if the running program is killed by an external process such as the \texttt{timeout} command, which means we are unable to give accurate figures for memory usage for benchmarks which did not complete within the allotted time.


\subsection{Host Graphs}
\label{subsec:hosts}

Host graphs used for benchmarking are named to give an indication of their overall structure.


\paragraph*{Linear $n$}

A chain of $n$ nodes. The first node has only a single outgoing edge. The last node has only a single incoming edge, and all other nodes have exactly one incoming and one outgoing edge.

\paragraph*{Cyclic $n$}

As Linear $n$, but with an extra edge linking the first and final placed nodes so that every node has exactly one incoming and one outgoing edge.

\paragraph*{$x \times y$ Grid}

A square lattice of $x$ nodes wide by $y$ nodes tall.

\paragraph*{Gen $n$}

The Sierpinski program expects a host graph containing a single node with a numeric label, which is used as the number of iterations of the \texttt{expand!} rule to run.


The \textit{shortest distances} benchmark requires additional information in the form of costs for traversing each edge. For all of the host graphs for this program, half of the edges were assigned a cost of one and half were assigned a cost of two.




\begin{table}[h]
\begin{minipage}{\textwidth}
\centering

\begin{tabular}{llrrcrr}
\hline 
&  & & & & \multicolumn{2}{c}{Heap/kB}\\
Benchmark          & Host Graph\footnotemark & Apps & Time/s   & & Allocd & Live \\
\hline 
Acyclicity test
 &             3x3 grid &    12 &    0.02 & &  2048 &   129 \\
 &             5x5 grid &    40 &    0.03 & &  3072 &   382 \\
 &             7x7 grid &    84 &    0.17 & &  4096 &  1119 \\
 &             9x9 grid &   144 &    0.70 & &  6144 &  2100 \\
 &           cyclic 100 &     0 &    0.04 & &  3072 &   778 \\
 &           cyclic 500 &     0 &    0.46 & & 14336 &  5646 \\
 &          cyclic 1000 &     0 &    1.76 & & 25600 & 10368 \\
\hline
Shortest distances
 &             5x5 grid &    38 & $<0.01$ & &  3072 &   414 \\
 &             7x7 grid &    90 &    0.08 & &  4096 &  1177 \\
 &             9x9 grid &   175 &    0.39 & &  8192 &  3172 \\
\hline
Sierpinski
 &                gen 2 &     7 & $<0.01$ & &  2048 &   133 \\
 &                gen 3 &    17 &    0.14 & &  5120 &  1056 \\
 &                gen 4 &    45 &    6.52 & & 58368 & 18313 \\
 &                gen 5 & - & $>5m$ & & - & - \\
\hline
Transitive closure
 &            linear 05 &     6 & $<0.01$ & &  2048 &   144 \\
 &            linear 10 &    36 &    0.04 & &  2048 &   144 \\
 &            linear 20 &   171 &    1.67 & & 21504 &  7073 \\
 &            linear 30 &   406 &   14.39 & & 103424 & 33152 \\
 &            linear 40 &   741 &   66.31 & & 324608 & 103275 \\
 &            linear 50 & - & $>5m$ & & - & - \\
\hline
Vertex colouring
 &             3x3 grid &    27 &    0.02 & &  2048 &   140 \\
 &             5x5 grid &   125 &    0.03 & &  3072 &   999 \\
 &             7x7 grid &   343 &    0.17 & &  9216 &  3681 \\
 &             9x9 grid &   729 &    0.89 & & 25600 & 11438 \\
\hline

\end{tabular}

\caption[Reference interpreter benchmarks]{Reference interpreter benchmark results when generating a single output graph}

\label{table:resultsSingle}
\end{minipage}
\end{table}



\begin{table}[h]
\begin{minipage}{\textwidth}
\centering

\begin{tabular}{llrrrrrcrr}
\hline 
&  & \multicolumn{3}{c}{Output Graphs} & & && \multicolumn{2}{c}{Heap/kB}\\
Benchmark          & Host Graph\footnotemark & Total & Unique   & Failed & Apps & Time/s   & & Total  & Live \\
\hline 
Acyclicity test
 &             2x2 grid &      6 &         1 &     0 &     4 & $<0.01$ & &  2048 &   134 \\
 &             3x3 grid &  19770 &         1 &     0 &    12 &   12.00 & & 10240 &  3301 \\
 &             4x4 grid & - & - & - & - & $>5m$ & & - & - \\
 &           cyclic 100 &      0 &         0 &   100 &     0 &    0.06 & &  4096 &   784 \\
 &           cyclic 500 &      0 &         0 &   500 &     0 &    0.86 & & 14336 &  5651 \\
 &          cyclic 1000 &      0 &         0 &  1000 &     0 &    3.31 & & 26624 & 11053 \\
\hline
Shortest distances
 &             2x2 grid &      6 &         1 &     0 &     4 & $<0.01$ & &  2048 &   131 \\
 &             3x3 grid &  28924 &         1 &     0 &  9-14 &   19.15 & & 167936 & 58180 \\
 &             4x4 grid & - & - & - & - & $>5m$ & & - & - \\
\hline
Sierpinski
 &                gen 2 &      6 &         1 &     0 &     7 &    0.04 & &  3072 &   242 \\
 &                gen 3 & - & - & - & - & $>5m$ & & - & - \\
\hline
Transitive closure
 &            linear 05 &    866 &         1 &     0 &     6 &    0.44 & &  6144 &  1699 \\
 &            linear 10 & - & - & - & - & $>5m$ & & - & - \\
\hline
Vertex colouring
 &             2x2 grid &    480 &         2 &     0 &   6-8 &    0.07 & &  5120 &  1598 \\
 &             3x3 grid & - & - & - & - & $>5m$ & & - & - \\
\hline

\end{tabular}

\caption[Reference interpreter benchmarks]{Reference interpreter benchmark results when generating all possible output graphs}

\label{table:resultsAll}
\end{minipage}
\end{table}


\subsection{Benchmark performance}

Tables \ref{table:resultsSingle} and \ref{table:resultsAll} summarise the reference interpreter performance for the five benchmark programs, each run on several host graphs. Table \ref{table:resultsAll} shows figures for generating all possible results for each program/host-graph pair, while table \ref{table:resultsSingle} shows results when the interpreter is running in single output graph mode.



Comparing tables \ref{table:resultsAll} and \ref{table:resultsSingle} gives an indication of the additional costs associated with generating all graphs and isomorphism-checking the results.



% Generating all possible intermediate graphs

\subsubsection*{Exhaustive search}

When viewing the results in \ref{table:resultsAll}, it should be remembered that the interpreter is computing \textit{all possible} output graphs for a given program/host-graph pair, then consolidating the results based on isomorphism. It is therefore unsurprising that execution time increases exponentially with increasing size of host graph.

However, as already mentioned, more consideration was given to code-correctness and readability than to execution speed or memory consumption.

While performance is acceptable for some quite complex programs, it is nevertheless easy to find cases in which performance rapidly degrades to an unacceptable degree. A particular example being the vertex-colouring benchmark program, which exhibits exponential growth in the number of possible intermediate graphs with increasing edge counts.

One of the most serious limits on the performance of the reference interpreter, which impacts matching of any non-trivial rule, is the naive node-matching strategy. Given a rule with $n$ nodes, select all permutations of $n$ nodes from the host graph, then discard those that fail to meet rootedness, label and edge constraints.

While from the point of view of simplicity this is a reasonable choice for finding all possible outputs for each rule, it is far from an optimal way of finding a single output graph; for any non-trivial rule and host pair the chances of $n$ nodes selected essentially at random matching a rule is very low, meaning that many permutations need to be inspected for each successful match.

That this strategy is repeated for all of the intermediate graphs generated by the previous rule application explains the combinatorial explosions we see in heavily non-deterministic programs like \textit{Sierpinski} and \textit{Shortest distances}.

% interpreter cost vs compiler
\subsubsection*{Interpreter costs}

Any interpreter has run-time costs which would be paid at compile-time in a compiler. These costs can account for a significant percentage of the execution time.

In the sub-second execution time benchmarks, profiling information indicated that a significant portion of the execution time was spent parsing the rule-graphs and building map structures.


\subsubsection*{Separate graph and edge matching}

As discussed in section \ref{sec:graph-match} the reference interpreter matches nodes and edges in separate passes. This makes for a simple algorithm at the expense of performance.

A more performance focussed implementation would use a \textit{search plan}\cite{Horvath-Varro07} in which a graph morphism is built incrementally by adding nodes and edges to an existing partial morphism, and back-tracking if no suitable graph elements could be found.


\subsubsection*{Generic data structures}

The current version of the interpreter uses a generic lazy map data structure which gives a reasonable balance between cost of search and cost of update. 


Our first prototypes stored the graphs as linked-lists of key/value pairs. The result was an interpreter which spent most of its execution time traversing these lists retrieving nodes and edges. Switching to the faster map structure alone yielded a factor of two speed improvement.

Nevertheless, an underlying data structure tuned to our specific usage patterns and judiciously indexed could dramatically reduce this high cost of node and edge retrieval.


\subsubsection*{Isomorphism checking}

Another cost associated with generating all possible output graphs is isomorphism checking. In benchmarks where nondeterminism gives us multiple ways of reaching similar output graphs we want a way of reducing this potentially large set of outputs to those which are truly distinct. A particularly extreme example of this can be seen in the \textit{3x3 grid} host graph of the \textit{Shortest distances} benchmark. Here we produced nearly 30~000 isomorphic output graphs! Presumably if allowed to run to completion the \textit{4x4 grid} would yield several orders of magnitude more.


% Cost of profiling
\subsubsection*{Profiling costs}

We have also incurred some incidental costs by gathering the profiling information used to generate 

% TODO: guestimate profiling costs


% TODO: fail early and fail cheaply!
