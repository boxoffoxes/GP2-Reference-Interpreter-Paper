\section{Graph Programs}
\label{sec:graph-programs}

We briefly introduce GP by discussing two example programs (see \cite{Plump12a} for a language definition). The principal programming construct in GP are conditional graph transformation rules labelled with expressions. 


\begin{example}[Transitive Closure]

\begin{tabular}{lp{10.5cm}}
\ul{Input:} & Any graph $G$. \\
\ul{Output:} & The smallest extension of $G$ with unlabelled edges making the graph transitive (where ``unlabelled'' edges are labelled with the empty list).
\end{tabular}
  
\begin{figure}[htb]
\begin{center}
 \input{Programs/trans_closure.prog}
\end{center}
%\vspace{-.5\baselineskip}
\caption{Program for transitive closure}\label{fig:transitive-closure}
\end{figure}

\ul{Notes}
\begin{enumerate}
\setlength{\itemsep}{-.5ex}
\item The output is unique up to isomorphism.
\item The derivation length is bounded by $|V_G|^2$.
\end{enumerate}
\end{example}


\begin{example}[Vertex Colouring]

\begin{tabular}{lp{10.5cm}}
\ul{Input:} & Any graph $G$. \\
\ul{Output:} & The graph obtained from $G$\/ by marking grey each node and replacing its label $l$\/ with $l{:}i$, where $1 \leq i \leq |V_G|$ is the node's \emph{colour}. The source and target of each non-loop edge have different colours.
\end{tabular}
  
\begin{figure}[htb]
\begin{center}
 \input{Programs/vertex-colouring.prog}
\end{center}
%\vspace{-.5\baselineskip}
\caption{Program for vertex colouring}\label{fig:vertex-colouring}
\end{figure}

\ul{Notes}
\begin{enumerate}
\setlength{\itemsep}{-.5ex}
\item The derivation length is bounded by $|V_G|^2$.
\item The program is highly non-deterministic. For the same input graph, it may return outputs with different numbers of colours.
\end{enumerate}
\end{example}

